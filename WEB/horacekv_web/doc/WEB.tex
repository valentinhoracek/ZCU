\documentclass[czech,SP]{thesiskiv}

\author{Valentin Horáček}
\declarationmale
\title{Webová aplikace konferenčního systému}

\usepackage[utf8]{inputenc}
\usepackage[IL2]{fontenc}
\usepackage[czech]{babel}

\usepackage{graphicx}

\begin{document}
\maketitle
\tableofcontents
%
%	Chapter
%
\chapter{Zadání}
\par
Vaším úkolem bude vytvořit webové stránky konference.  Téma konference si můžete zvolit libovolné.
\par
Uživateli systému jsou autoři příspěvků (vkládají abstrakty článků a PDF dokumenty), recenzenti příspěvků 
(hodnotí příspěvky) a administrátoři (spravují uživatele, přiřazují příspěvky recenzentům k hodnocení a 
rozhodují o publikování příspěvků). Každý uživatel se do systému přihlašuje prostřednictvím vlastního 
uživatelského jména a hesla. Nepřihlášený uživatel vidí pouze publikované příspěvky.
\par
Nový uživatel se může do systému zaregistrovat, čímž získá status autora.
\par
Přihlášený autor vidí svoje příspěvky a stav, ve kterém se nacházejí (v recenzním řízení 
/ přijat + hodnocení / odmítnut + hodnocení). Své příspěvky může přidávat, editovat a volitelně i mazat. 
Rozhodnutí, zda autor může editovat či mazat publikované příspěvky je ponecháno na tvůrci systému.
\par
Přihlášený recenzent vidí příspěvky, které mu byly přiděleny k recenzi, a může je hodnotit 
(nutně alespoň 3 kritéria hodnocení). Pokud příspěvek nebyl dosud publikován, tak své hodnocení může změnit.
\par
Přihlášený administrátor spravuje uživatele (určuje jejich role a může uživatele zablokovat či smazat), 
přiřazuje neschválené příspěvky recenzentům k hodnocení (každý příspěvek bude recenzován minimálně třemi 
recenzenty) a na základě recenzí rozhoduje o publikování nebo odmítnutí příspěvku. 
Publikované příspěvky jsou automaticky zobrazovány ve veřejné části webu.
\par
Databáze musí obsahovat alespoň 3 tabulky, které budou dostatečně naplněny daty tak, aby bylo možné 
předvést funkčnost aplikace.

%
%	Chapter
%
\chapter{Použité technologie}
\section{HTML 5}
\par
HyperText Markup Language je jazyk, který se běžně používá pro tvorbu webových stránek. Jazyk pomocí svých
značek definuje jaký mají jednotlivé stránky vzhled.

\section{PHP}
\par
Tento skriptovací programovací jazyk se využívá pro dynamické webové stránky a zejména pro čtení vstupů z formůlářů, tedy
vstupů od uživatelů. Jednotlivé skripty jsou zpracovávány na straně webového serveru a k uživateli je přenesen
pouze výsledek jejich činnosti. 

\section{SQL}
\par
SQL je strukturovaný dotazovací jazyk pro práci s daty v relačních databázích. Pomocí SQL tedy pomocí jednotlivých dotazů jsem
získával veškerá potřebná data z databáze.

\section{Bootstrap}
\par
Bootstrap je knihovna nástrojů, která obsahuje přednastavené komponenty ať už pomocí CSS nebo JavaScriptu.
Díky tomu jsem webové aplikace responzivní a dynamický vzhled.

%
%	Chapter
%
\chapter{Adresářová struktura}
\par
Adresářová struktura webové aplikace je rozdělena, aby odpovídala architektuře MVC. Mimo tuto strukturu jsou pouze soubory
\textbf{index.php}, \textbf{settings.inc.php} a \textbf{.htaccess}, které splňují svůj účel v kořenovém adresáři.

\section{controllers}
\par
Složka obsahující veškeré kontrolery, tedy řídící část webové aplikace.
\section{doc}
\par
Zde je uložena dokumentace semestrální práce.
\section{models}
\par
Obsahuje soubor s funkcemi pro práci s databází a dva soubory pro vytvoření a naplnění databáze testovacími daty.
\section{uploads}
\par
V této složce jsou uloženy soubory, které uživatelé nahrají při nahrávání článku.
\section{views}
\par
Pro každý kontroler je zde pohled, ve kterém je naprogramováno, jak má stránka vypadat. Kromě daných pohledů jsou zde navíc soubory
\textbf{basic.phtmp} a \textbf{structure.phtmp} obsahující funkce pro hlavičku a patičku a obecnou strukturu webové aplikace.

%
%	Chapter
%
\chapter{Struktura databáze}
Databáze obsahuje 3 tabulky - \textbf{users}, \textbf{articles} a \textbf{reviews}, všechny s prefixem \textbf{horacekv\textunderscore XXX}.
\section{users}
Tabulka registrovaných uživatelů webové aplikace.
\begin{itemize}
\item \textbf{ID\textunderscore USER} - Číselný identifikátor uživatele
\item \textbf{FULL\textunderscore NAME} - Celé jméno uživatele
\item \textbf{LOGIN} - Přihlašovací jméno uživatele
\item \textbf{PASSWORD} - Heslo uživatele zakódováno pomocí \textbf{md5}
\item \textbf{EMAIL} - Email uživatele
\item \textbf{ROLE} - Role uživatele: \textbf{AUTHOR}, \textbf{REVIEWER}, \textbf{ADMIN}
\end{itemize}

\section{articles}
V této tabulce jsou informace o jednotlivých článcích.
\begin{itemize}
\item \textbf{ID\textunderscore ARTICLE} - Číselný identifikátor článku
\item \textbf{TITLE} - Název článku
\item \textbf{FILE\textunderscore NAME} - Název nahraného souboru
\item \textbf{FILE\textunderscore DATA} - Data nahraného souboru
\item \textbf{ABSTRACT} - Abstrakt článku
\item \textbf{REVIEWS} - Počet recenzí pro tento článek
\item \textbf{STATE} - Stav tohoto článku: \textbf{UNDER REVIEW}, \textbf{ACCEPTED}, \textbf{DECLINED}
\item \textbf{ID\textunderscore AUTHOR} - Číselný identifikátor autora článku
\end{itemize}

\section{reviews}
Zde je uložen záznam každé recenze článku.

\begin{itemize}
\item \textbf{ID\textunderscore REVIEW} - Číselný identifikátor recenze
\item \textbf{ID\textunderscore REVIEWER} - Číselný identifikátor recenzenta
\item \textbf{ID\textunderscore ARTICLE} - Číselný identifikátor článku
\item \textbf{SCORE\textunderscore 1} - První hodnocení
\item \textbf{SCORE\textunderscore 2} - Druhé hodnocení
\item \textbf{SCORE\textunderscore 3} - Třetí hodnocení
\end{itemize}

%
%	Chapter
%
\chapter{Uživatelská příručka}

\section{Nepřihlášený uživatel}
Nepřihlášený uživatel má k dispozi pouze hlavní stránku, stránku s informacemi, možnost se registrovat či přihlásit
a prohlížení schválených článků.
\subsection{Registrace}
Z hlavní nabídky vyberte možnost \textbf{Account-Registration}. Na nové stránce vyplňte všechny potřebné informace
a zvolte tlačítko \textbf{Register}. Webová aplikace Vás automaticky přihlásí.
\subsection{Přihlášení}
Z nabídky zvolte možnost \textbf{Account-Login}. Nyní zadejte zvolené přihlašovací jméno a heslo.
\subsection{Prohlížení článků}
Z hlavního menu vybrte odkaz \textbf{Articles}. Na nové stránce jsou zobrazeny kategorie článků. Pro nepřihlášené uživatele
jsou vidět pouze schválené články.
\section{Autor}
Status autora dostane každý nový uživatel po registraci.
\subsection{Odhlášení}
Z hlavní nabídky rozklikněte Vaše uživatelské jméno po kliknutí na možnost \textbf{Logout} Vás systém odhlásí.
\subsection{Úprava účtu}
Pod Vaším uživatelským jménem je také možnost \textbf{Account}. Pokud ji zvolíte můžete upravovat všechny svoje informace.
\subsection{Nahrání článku}
Možnost \textbf{Editor} v hlavní nabídce umožňuje přihlášeným uživatelům vytvořit nový článek. Je nutné vyplnit všechny pole
a nahrát PDF soubor.
\subsection{Úprava článku}
Pokud máte vytvořen vlastní článek zobrazí se Vám pod možností \textbf{Articles} z hlavní nabídky. Pokud vyberete svůj článek a rozkliknete jeho název
získáte možnost ho upravit nebo smazat.
\subsection{Detail hodnocení článku}
Z nabídky svých článku se také můžete podívat na hodnocení vašich článků. Stačí kliknout v tabulce na počet recenzí u vašeho článku
a zobrazí se Vám nová tabulka se jmény recenzentů a hodnocením zvoleného článku.
\section{Recenzent}
Pokud se stanene recenzenty přibudou Vám nové možnosti.
\subsection{Hodnocení článku}
Jste-li vybráni k recenzi nějakého článku zobrazí se Vám tyto články jako nová kategorie na stránce \textbf{Articles}. 
Vyberte článek a budete přesměrováni na novou stránku,na které můžete vybrat svoje hodnocení. Jakmile své hodnocení odešlete nebude ho možné změnit.
\section{Admin}

\subsection{Úprava uživatelů}
Administrátorům se v hlavní nabídce zobrazuje možnost \textbf{Administration}, která zobrazuje všechny uživatele dle jejich práv v aplikaci.
Po vybrání jednoho z uživatelů lze uživatele vymazat nebo změnit jejich práva.
\subsection{Zadání recenze}
Pro administrátory se také objeví nová kategorie článků. Jedná se o články, které ještě neprošly recenzí nebo recenzí není dostatečný počet.
Po vybrání článku z této kategorie lze určit recenzenta z nabídky.

%
%	Chapter
%
\chapter{Závěr}
\par
Semestrální práce splňuje zadání v celém rozsahu. Webová aplikace asi není ihned použitelný web, je potřeba doplnit konkrétní informace o konferenci a
o kritériích hodnocení článku.
Díky využívání nástroje Bootstrap nebylo potřeba vytvářet nové styly a třídy, ale stačilo využít nabízené možnosti pro vytvoření
jednoduchého responzivního designu.

\end{document}